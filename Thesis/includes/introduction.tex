\indent	V dnešnej dobe je svet každým dňom stále väčšmi digitalizovaný. Neustále vznikajú nové produkty a služby. Systémy, ktoré nám pomáhajú mať všetko na jednom mieste. Bezpečnosť týchto systémov je rovnako dôležitá, ako jej funkčnosť. Keďže uchovávajú citlivé informácie používateľov, je absolútne kľúčové ich chrániť pred útokmi. Preto veľká väčšina aplikácií a systémov, ktoré pracujú s informáciami, používa účty. Používateľ si vytvorí svoj účet a dostane sa doň pomocou dvoch vstupov: používateľského mena a hesla. Heslo jeho účet chráni, keďže používateľské meno je verejné.
	\par Na začiatku som uvádzal, že neustále vznikajú nové produkty, služby, či systémy. Môžeme teda očakávať, že bežný človek ich bude využívať viacero na dennej báze. Povedzme, že používa email, má účty v niekoľkých sociálnych sieťach, používa aplikáciu na elektronické bankovníctvo, je zaregistrovaný v niekoľkých internetových obchodoch, pravidelne pristupuje k svojim dátam na cloude (online úložisko) a podobne. Každá z týchto položiek pracuje s nejakým heslom, ktorá autentifikuje osobu, ktorá heslo zadala.
	\par Ak hovoríme o hesle ako o reťazci, teda postupnosti znakov, používateľ si tento reťazec musí pamätať, aby mohol vstúpiť do systému. Tu vzniká problém. Problém pamätania si každého hesla pre každú aplikáciu. Existujú dve riešenia. Prvou možnosťou je nastavenie ľahko zapamätateľného, prípadne rovnakého hesla do všetkých účtov. Týmto sa dramaticky znižuje úroveň bezpečnosti. Zároveň sa ale zvyšuje level komfortu pri interakcii s aplikáciami.
	\par Tou druhou možnosťou je použitie aplikácie typu password manager (správca hesiel). Môžeme o ňom uvažovať ako o trezore. Dovnútra môžeme uložiť všetky naše heslá a zamknúť ich pod jedným kľúčom. Situácia sa odrazu mení. Zrazu si nemusíme pamätať niekoľko hesiel, ale iba jedno. Úroveň komfortu pri interakcii s aplikáciami ostáva zachovaná, avšak rovnako je dosiahnutá vysoká úroveň bezpečnosti.
	\par V praxi vytvára implementácia tohto managera určitý ekosystém. Teda, password manager dokáže poskytovať svoje služby len zariadeniu, na ktorom je nainštalovaný. Mimo tohto prostredia používateľ stráca znalosť o svojich údajoch. V súčasnosti považujeme oblasť ekosystému password managera za dostatočne rozvinutú. Preto sme sa rozhodli venovať oblasti mimo neho.
	\par Cieľom práce je vyvinúť aplikáciu password manager a nájsť bezpečný, pohodlný a efektívny spôsob prístupu používateľa k heslám na cudzom zariadení. Teda na takom zariadení, kde password manager s citlivými údajmi používateľa nie je prítomný (mimo ekosystému). Snahou bude vymyslieť riešenie (riešenia), ktoré by toto umožňovali.