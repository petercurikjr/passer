\indent Počas celej práce sme sa snažili nájsť bezpečný, pohodlný a efektívny spôsob prístupu používateľa k heslám na cudzom zariadení. Teda na takom zariadení, kde password manager s citlivými údajmi používateľa nie je prítomný (mimo ekosystému).

Nami nájdený spôsob je pohodlný. Používateľ nemusí na cudzom zariadení nič inštalovať. Stránka nemá vysoké nároky na výkon. Má minimum elementov, ktoré sa rýchlo načítajú aj na zariadeniach so slabším výkonom/pomalým pripojením na internet. 

Najpohodlnejší spôsob je verifikácia QR kódom. Ak má používateľ funkčnú kameru na svojom smartfóne, celý proces verifikácie sa vykoná do sekundy.

V opačnom prípade je pre neho k dispozícii šesťciferný kód. Pamätanie čísiel nie je náročné, ich zadávanie je rýchle. Po ich zadaní je verifikácia taká rýchla ako pri QR kóde. 

Snažili sme sa o to, aby bola aplikácia Passer prehľadná a ,,user-friendly''. Pri jednotlivých položkách má používateľ k dispozícii tlačidlo \textit{Outsider}, takže sa rýchlo dostane k prenosu dát. Z hľadiska zložitosti používania, či časovej zložitosti sme splnili požiadavky: riešenie je lepšie ako manuálne prepisovanie z password managera, či inštalácia podpornej aplikácii na cudzom zariadení, keď ide o niečo jednorázové.

Týmto sme zodpovedali aj otázku efektivity. Úkony (verifikácia) sú rýchle. Na serveri ukladáme dáta do cache, čo je najrýchlejší spôsob, ako k nim pristúpiť. Používame symetrickú kryptografiu. Tá je oveľa rýchlejšia ako asymetrická. 

Ako sme písali v \nameref{hrozby}, to, či je naše riešenie bezpečné, sa zistí časom. Ale z teoretického hľadiska sa nám podarilo ochrániť naše riešenie tak, aby uspokojilo dnešné bezpečnostné štandardy. AES, HTTPS, SHA-256, ECDH sú overené protokoly, algoritmy a štandardy, ktoré sú bezpečné a spoľahlivé. Preto sme ich implementovali do finálneho riešenia. 