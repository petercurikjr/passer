\subsection{Server}
Server prijíma a posiela dáta pomocou skriptu, ktorý je napísaný v jazyku Python. Konkrétne hovoríme o frameworku Flask.

Flask pre rôzne url endpointy vykoná inú funkciu. Základný komponent je anotácia \texttt{@app.route(`/pripona/url')} \cite{flask_approute}, kde jedným z parametrov do funkcie \texttt{route()} je reťazec, ktorý symbolizuje url endpoint, pre ktorý sa má funkcia vykonať. Definícia funkcie a jej telo nasledujú hneď za \texttt{@app.route()}. 

Pre ukážku povieme, že server funguje na doméne \texttt{https://flask.website.com}. Potom, ak existuje nasledujúci kód:
\newline
\begin{lstlisting}[language=Python, basicstyle=\small]
@app.route(`/test')
def function():
    currentDate = datetime.now().strftime('%d/%m/%Y %H:%M:%S')
    return `Hello World! ' + currentDate
\end{lstlisting}
\leavevmode\newline
Tak po zadaní url \texttt{https://flask.website.com/test} sa zobrazí text

\begin{center}
    \textit{Hello World! 26/5/2020 22:55:32}
\end{center}
(čas, keby sme zadali url v čase písania tohto textu).

Všetky dáta, ktoré server uchováva sú dočasné. Nachádzajú sa v dátovej cache. Každý záznam v nej je jedinečný a po jeho použití zmizne. Na implementáciu cache v Pythone budeme používať knižnicu Werkzeug \cite{werkzeug_cache}. Na začiatku si vytvoríme jej inštanciu \texttt{cache = SimpleCache()}. Následne ju môžeme používať. V kóde neskôr používame jej tri najčastejšie metódy a to: \texttt{cache.set()} (pri pridávaní nových dát), \texttt{cache.get()} (pri verifikácii) a \texttt{cache.delete()} (po úspešnej verifikácii). Každý záznam, ktorý vložíme do cache má dobu platnosti 2 minúty. Po tejto dobe je záznam automaticky zmazaný.
