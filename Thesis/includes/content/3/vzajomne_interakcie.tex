\subsection{Vzájomné interakcie}
\label{vzajomne_interakcie}

\subsubsection{Passer - Server (šesťciferný kód)}
Teraz ukážeme príklad, ako server spracováva HTTP POST request od aplikácie Passer. Aplikácia odošle nový, šesťciferný verifikačný kód spolu s heslami, ktoré používateľ cez Outsider odoslal. Inými slovami, pošle JSON štruktúru \texttt{SixdigitAuth} (spomíname v \nameref{passer}). Štruktúra môže vyzerať nasledovne: 
\newline
\lstset{literate={č}{{\v{c}}}1{ú}{{\'u}}1}
\begin{lstlisting}[basicstyle=\small]
{
    "deviceID":"F19494AE-555C-49FF-87DD-D94CA62D9634",
    "sixdigitCode":"791509",
    "passwordItems":[
        {
            "password":"password123",
            "id":"595C185D-DDF9-4BEA-A722-D383C75E5B33",
            "username":"username@example.com",
            "favourites":true,
            "group":1,
            "itemname":"Gmail účet",
            "url":"mail.google.com"
        },
    ],
    "bankCardItems":[],
    "otherItems":[],
    "timestamp":612273137.84391904
}
\end{lstlisting}
\leavevmode\newline
\indent V tejto ukážke si používateľ vybral možnosť verifikácie pomocou šesťciferného kódu a posiela jedno heslo - \textit{Gmail účet}. Server tieto dáta očakáva v tomto bloku kódu:
\newline
\begin{lstlisting}[language=Python, basicstyle=\small]
@app.route('/sixdigit', methods=['POST'])
def processSixDigitFromApp():
    incomingData = request.get_json()
    
    deviceID = incomingData['deviceID']
    sixdigitCode = incomingData['sixdigitCode']
    passwordItems = incomingData.get('passwordItems')
    bankCardItems = incomingData.get('bankCardItems')
    otherItems = incomingData.get('otherItems')
\end{lstlisting}
\leavevmode\newline
\indent Server si uloží jednotlivé atribúty JSON štruktúry do premenných. Všimnime si, že JSON atribút \texttt{timestamp} server vôbec nespracuje. Hovorili sme, že \texttt{timestamp} slúži len pre počítanie zostávajúceho času pre aplikáciu Passer. Tento čas je avšak orientačný. Medzi vytvorením inštancie \texttt{SixdigitAuth} (vtedy vzniká časový odtlačok) a zapísaním verifikácie do Flask cache existuje určité časové oneskorenie. Preto by bolo nepresné, aby server uvažoval \texttt{timestamp} ako čas do expirácie verifikácie. 

Po spracovaní JSON štruktúry nasleduje kontrola duplicít. Nasledujúci diagram ukazuje logiku ukladania dát do cache servera (pri šesťcifernom kóde): 
\newline
\begin{figure}[H]
  \centering
  \includegraphics[width=10cm]{img/cache-diagram.pdf}
  \caption{Ukladanie dát do cache servera (šesťciferný kód).}
  \label{cache-diagram}
\end{figure}

Šesťciferný kód je namapovaný na ID zariadenia, ktorý je kľúč. A následne daný kód slúži ako kľúč v druhom zázname, na ktorý sú namapované Passer položky používateľa.

Dôvod, prečo potrebujeme až dve mapovania vysvetlíme. Potrebujeme zaručiť, aby sa dvom rôznym používateľom (teda dvom rôznym ID zariadenia) nemohol vygenerovať rovnaký kód. Rovnako potrebujeme, aby jeden používateľ (teda jedno ID zariadenia) nemal na sebe viacero jednorázových kódov (spomíname v \nameref{passer}). 

Pred zápisom novej verifikácie do cache najprv skontrolujeme, či nenastal jeden z dvoch konfliktných stavov, ktoré sme práve opísali.

Ako prvé zistíme, či v cache existuje kľúč, ktorý by sa zhodoval so šesťciferným kódom, ktorý práve prišiel. Ak existuje, server vráti kód 409, čo znamená CONFLICT \cite{http_response}. Passer na to zareaguje opakovaním celej operácie. Čo obnáša zaslanie \texttt{SixdigitAuth} štruktúry v JSON formáte s novým kódom. 
\begin{lstlisting}[language=Python, basicstyle=\small]
    if cache.has(sixdigitCode):
        return 409
\end{lstlisting}

Druhá kontrola spočíva v tom, či pre daného používateľa už existuje nejaký šesťciferný kód. Kontrolujeme, či v cache existuje kľúč, ktorý by sa zhodoval s \texttt{deviceID}, teda s ID zariadenia. Ak existuje, musíme vymazať obe mapovania. 
\begin{lstlisting}[language=Python, basicstyle=\small]
    if cache.has(deviceID):
        userOldVerifData = cache.get(deviceID)
        cache.delete(deviceID)
        cache.delete(userOldVerifData)
\end{lstlisting}

Po týchto operáciách môže prebehnúť samotný zápis do cache. Pomocou \texttt{cache.set()} nastavíme mapovanie kľúč:hodnota a povieme, ako dlho má tento záznam existovať. Obe mapovania (\texttt{deviceID:sixdigitCode} a \texttt{sixdigitCode:}Passer položky) budú trvať po dobu dvoch minút.
\begin{lstlisting}[language=Python, basicstyle=\small]
    cache.set(deviceID,sixdigitCode,timeout=2*60)
    cache.set(sixdigitCode,[passwordItems, bankCardItems, otherItems, deviceID],timeout=2*60)
    return 'server: ok', 201
\end{lstlisting}

Po úspešnej operácii Passer zobrazí používateľovi šesťciferný kód so zostávajúcim časom platnosti a inštrukciami.
\subsubsection{Webstránka - Server (šesťciferný kód)}
Používateľ už obdržal kód, ktorý môže použiť na webstránke na prístup k svojim položkám. Po jeho zadaní sa v JavaScript kóde webstránky spustí HTTP POST request na server. Posiela mu kód zadaný používateľom a čaká, ako odpovie server. Server sa snaží získať dáta namapované ku kľúču (\texttt{data = cache.get(sixdigitTyped)}). Kľúč je používateľom zadaný šesťciferný kód. 

Ak by kód nebol správne zadaný, kľúč v cache by sa nenašiel. Tým pádom by \texttt{data} boli \texttt{None}. Podľa toho sa mení správanie kódu nižšie: 
\begin{lstlisting}[language=Python, basicstyle=\small]
    if data != None:
        cache.delete(sixdigitTyped)
        cache.delete(data[-1])
        response[sixdigitTyped] = data
        return response
    return 'Wrong code'
\end{lstlisting}

Predtým, než vráti dáta späť webstránke, server vymaže záznamy v cache. Najskôr záznam pod kľúčom šesťciferného kódu, potom záznam pod kľúčom ID zariadenia (posledný člen zoznamu \texttt{data}). Následne vráti dáta.

Webstránka presmeruje používateľa na podstránku \texttt{.../passwords.html} a to nasledovne: \texttt{window.location.replace("passwords.html")}. Predtým ale potrebuje určitým spôsobom odovzdať dáta, ktoré práve od servera získala podstránke. JavaScript ponúka mnoho riešení, jedno z nich je local storage \cite{localstorage}.

\begin{sloppypar}
    Ide o schopnosť ukladať dáta v tvare kľúč:hodnota (podobné, ako cache na serveri) lokálne v prehliadači. Pomocou \texttt{window.localStorage.setItem("serverData",xhr.responseText)} uložíme do local storage odpoveď zo servera pod kľúč \texttt{serverData}. Následne sme schopní dáta z local storage vybrať \texttt{const serverData = window.localStorage.getItem("serverData")
    } a pracovať s nimi. Hneď po načítaní do premennej \texttt{serverData} je potrebné vyčistiť local storage, inak by boli tieto dáta v pamäti aj počas ďalších sessions, ako uvádza \cite{localstorage}. Jednoduchým kódom \texttt{localStorage.clear()} je náš problém vyriešený.
\end{sloppypar}

\subsubsection{Passer - Server - Webstránka (QR kód)}
Ako sme spomínali v sekcii \nameref{passer}, verifikácia pomocou QR kódu je špecifická. Počas nej prebieha interkacia so všetkými troma komponentmi (v úvodzovkách) naraz. 

Hneď po načítaní webstránky sa vygeneruje takzvaná session id: \texttt{let sessionID = Date.now().toString(36) + Math.random().toString(36).substr(2, 5)}. JavaScript používa aktuálny čas pre dosiahnutie jedinečnosti. To nemusí byť dostačujúce, preto sa k výslednému reťazcu pripája ešte náhodný reťazec pomocou funkcie \texttt{Math.random()}. Táto session id sa zakóduje do vygenerovaného QR kódu. Je zrejmé, že QR kód je zakaždým jedinečný, nakoľko obsahuje jedinečný session id reťazec.

Generovanie kódu je vďaka knižnici QRCode.js nasledovné: 
\newline
\begin{lstlisting}[language=JavaScript, basicstyle=\small]
function generateQRcode() {
    var qr = new QRCode(document.getElementById("qr-image"), {
        width: 120,
        height: 120,
        correctLevel : QRCode.CorrectLevel.L
    })
    qr.makeCode(sessionID)
}
\end{lstlisting}
\leavevmode\newline
\noindent V poslednom riadku si môžme všimnúť, že vytvárame QR kód s príslušnými dátami, teda s našim session id. To znamená, že ak niekto náš QR kód naskenuje, získa toto session id.

Tento bod je kľúčový. Pomocou session id sa vie používateľ s Passerom verifikovať, že je to práve on, kto kód naskenoval. Po naskenovaní v Passeri aplikácia vytvorí štruktúru, podobnú \texttt{SixdigitAuth}. Vyzerá takto:
\newline
\begin{lstlisting}[language=Swift, basicstyle=\small]
struct QRAuth: Codable {
    let sessionID: String
    let passwordItems: [PasswordItem]?
    let bankCardItems: [BankCardItem]?
    let otherItems: [OtherItem]?
}
\end{lstlisting}
\leavevmode\newline
\noindent \texttt{sessionID} je session id naskenovaná z QR kódu. K nemu pridáme polia Passer položiek, ktoré sa spolu so \texttt{sessionID} odošlú na server rovnakým spôsobom, ako pri šesťcifernom kóde. Pri QR verifikácii je len jedno mapovanie v cache servera: 
\newline
\begin{figure}[H]
  \centering
  \includegraphics[width=10cm]{img/cache-diagramQR.pdf}
  \caption{Ukladanie dát do cache servera (QR kód).}
  \label{cache-diagramQR}
\end{figure}

Medzičasom webstránka cyklicky kontroluje server, či už existuje v cache kľúč s daným session id. Aby sa zabránilo preťaženiu, robí tak iba raz za sekundu. Akonáhle webstránka dostane pozitívnu odpoveď a dáta používateľa, prestane kontrolovať server a presmeruje používateľa na podstránku \texttt{.../passwords.html}. Kontrola prítomnosti verifikácie v cache je založená na rovnakom princípe ako pri šesťcifernom kóde, takže hlbšiu analýzu tu nie je potrebné robiť.

Ukážeme ešte spôsob kontrolovania servera na strane webstránky:
\newline
\begin{lstlisting}[language=JavaScript, basicstyle=\small]
setInterval(function() {
    let xhr = new XMLHttpRequest()
    xhr.open("POST", "https://api-passer.herokuapp.com/verifyQRfromwebsite", true)
    xhr.setRequestHeader("Content-Type", "application/json")
    xhr.onreadystatechange = function () {
        if (xhr.readyState === 4 && xhr.status === 201) {
            window.localStorage.setItem("serverData",xhr.responseText)
            window.location.replace("passwords.html")
        }
    }
    let data = { "sessionID": sessionID }
    let jsonData = JSON.stringify(data)
    xhr.send(jsonData)
}, 1000) 
\end{lstlisting}
\leavevmode\newline
\indent Celý \texttt{XMLHTTPRequest()} objekt sa spúšťa až exekúciou riadku \texttt{xhr.send()} v dolnej časti kódu. Podobná logika requestov sa vykonáva aj v jazyku Swift, na ktorom je postavený Passer.

\leavevmode\newline
\indent Toto sú všetky interakcie našich troch komponentov, ktoré zhmotnili ideu celej práce. Týmto spôsobom sme docielili prístup používateľa k svojim dátam aj mimo Passera.