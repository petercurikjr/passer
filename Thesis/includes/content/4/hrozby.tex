\subsection{Zostávajúce hrozby}
\label{hrozby}
V našej implementácii sa spoliehame na Secure Enclave. Ak by sa útočník vedel nabúrať do SE, dostal by prístup ku všetkým kľúčom vo vnútri. A keby útočník prelomil SE iPhone-u, ktorý má nainštalovaný Passer, útočník by vedel dešifrovať všetky Passer dáta používateľa.

Ak by chcel útočník ukradnúť dáta v cache servera, musel by nájsť stroj, na ktorom je spustený skript. Z daného stroja by musel vybrať pamäť RAM a zmraziť ju tekutým dusíkom (Cold boot attack \cite{coldboot}). Tak by dáta v čase zmrazenia RAM v nej ostali po ľubovoľne dlhú dobu (pri optimálnych teplotných podmienkach). 

Iný spôsob, ako ukradnúť dáta je brute-force útok. Na webstránke má útočník dve minúty na to, aby spomedzi milión možností trafil konkrétny, aktívny, šesťciferný kód. Avšak naša webstránka vytvára oneskorenie pred ďalším pokusom písania. Väčšie nebezpečenstvo predstavuje botnet. Botnet je skupina počítačov, ktoré plnia príkazy ich ,,veliteľa'' (v našom prípade útočník). Uvažujme, že by mal botnet 1000 strojov. Každý z nich by nepretržite, automatizovane zadával rôzne šesťciferné kódy. Mal by otvorených 10 okien prehliadača, takže by paralelne skúšal 10 pokusov naraz. S našim oneskorením na webstránke za 2 minúty stíha jeden stroj v botnete 400 pokusov. Ak je v botnete 1000 takýchto strojov, za 2 minúty stihnú neuveriteľných 400 000 pokusov. To je pokrytie takmer polovice priestoru, do ktorého sa útočník snaží trafiť. 

Celkom závažný problém je moment, kedy sa používateľ na webstránke dostane ku svojim údajom. Potrebuje si ich skopírovať, aby ich vedel použiť. Pointa je, že tento úkon je cielený na cudzie zariadenia. To znamená, že ak používateľ dokončí svoju prácu a odíde od cudzieho zariadenia, je viac než pravdepodobné, že jeden zo svojich citlivých údajov nechá skopírovaný v schránke. Potom stačí, aby útočník počkal, kým obeť odíde. Po jej odchode príde k zariadeniu. Získa reťazec uložený v schránke a ukradne ho. Na tento problém sme nenašli riešenie.

Samotné cudzie zariadenie mimo ekosystému môže obsahovať malware. Konkrétne keylogger dokáže odchytávať písaný text do klávesnice, aj obsah schránky. Ak by si používateľ z webstránky skopíroval heslo, keylogger tieto dáta ukradne. Táto hrozba platí pri používaní cudzieho zariadenia všeobecne. Nie je viazaná na naše riešenie.


