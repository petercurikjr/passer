\subsection{Server a webstránka}
Pri týchto komponentoch je dôležité, aby pri výmene informácií bol protokol HTTP zabezpečený a Eva\footnote{pri spomínanom Bobovi a Alici je Eva chápaná ako útočník, ktorý chce spojenie narušiť alebo čerpať z neho citlivé informácie} nevedela kanál čítať. Z toho dôvodu sú oba komponenty zabezpečené HTTPS protokolom, ktorý zaobaľuje vrstvu HTTP do TLS vrstvy. Tá šifruje informácie, ktoré cestujú z Passera na server, zo servera na webstránku, z webstránky na server.

Server nemá žiadne dáta vo filesystéme. Teda, narozdiel od Passera nemusíme riešiť problematiku ukladania a výmeny kľúčov. Jediné dáta, ktoré server uchováva sú v cache. Tie musia byť v plaintexte. Z hľadiska bezpečnosti sa spoliehame na správnu konfiguráciu servera, jeho operačného systému a web servera, na ktorom je spustený náš Python skript.

Webstránka je na tom podobne. Prijaté dáta od servera po úspešnej verifikácii používateľom sú rovnako v plaintexte. Používateľ musí byť schopný ich skopírovať. Šifrovaný text je pre neho nepoužiteľný.

Hneď po získaní dát od servera vyčistíme local storage webstránky. Inak by boli dáta dohľadateľné, pokiaľ je otvorený prehliadač na cudzom zariadení.