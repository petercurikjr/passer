\subsection{Nízka popularita password managerov a nesprávne alternatívne spôsoby ukladania hesiel}
Ľahko sme vedeli ukázať, že bohatosť a rôznorodosť trhu s aplikáciami na správu hesiel je naozaj veľká. Používateľ by teda nemal mať problém vybrať si takú aplikáciu, ktorá vyhovuje jeho požiadavkám. Napriek tomu je popularita password managerov nízka. 
\par Už nadpis článku \cite{19} začína slovami \textit{,,Hardly Anybody Uses a Password Manager''}. Teda v slovenskom znení - ,,Horko-ťažko niekto vôbec používa password manager''. Píše o prieskume na túto tému a jeho výsledkoch. V Spojených štátoch a v Anglicku sa pýtali 1000 respondentov rôzne otázky. Týkali sa bezpečnosti pri používaní internetu. Išlo najmä o praktiky pri používaní hesiel; či pre každú stránku používajú iné heslo, či je heslo silné, dlhé alebo krátke, ,,náhodné'' alebo ľahko zapamätateľné. Spomedzi opýtaných, 59\% uviedlo, že používa 5 alebo menej rôznych hesiel, ktoré si pamätajú v hlave. Pritom 74\% sa denne prihlási do 6 a viac účtov. Môžeme už teraz hovoriť o pravdepodobnosti faktu, že niektorí z týchto používateľov používajú rovnaké heslo pre rôzne účty. Túto skutočnosť využívajú útočníci. Ak získajú heslo k jednému účtu, predpokladajú, že ho používateľ využíva aj v iných účtoch. Najmä, ak je heslo jednoduché. A ak hovoríme o respondentoch, ktorí si dokážu pamätať až do 5 hesiel v hlave a používajú ich na prihlasovanie do aspoň šiestich účtov, je veľmi pravdepodobné, že nie sú náročné (ani z hľadiska zapamätateľnosti, ani z hľadiska kryptografie).
\par Ďalšia negatívna skutočnosť, ktorá vyplýva z prieskumu je písanie hesiel na papier. Až 42\% opýtaných používa túto metódu. Niekto by argumentoval, že rovnako ako papier s heslami vie niekto ukradnúť aj server, kde sú uložené heslá password manageru. Aj keď sa môže zdať, že princíp je ten istý, na rozdiel od papiera s ručne vypísanými heslami sú heslá na serveroch password managerov zašifrované. Dnešné moderné šifry, ktoré sa na ich zašifrovanie používajú sú matematicky silné a v súčasnosti neprelomiteľné. Tá ťažšia časť je ukryť kľúč tak, aby ho nikto nenašiel. Dôvodom úspešných útokov teda nie je nedokonalosť moderných šifier. Väčšinou ide o postranné kanály, zlé ukrytie kľúča, spiknutie zvnútra a podobne.
\par Keď máme hovoriť o popularite password managerov, tento prieskum silno podporuje nadpis tejto podkapitoly. Len 8\% opýtaných používa na svoje heslá nejakého správcu. Naopak, jemne menej než tri štvrtiny si vystačí s tým, že si ich heslá zapamätá prehliadač. Tu treba uviesť skutočnosť, že ukladanie hesiel do prehliadača je náchylné na krádež pomocou malvéru (škodlivý softvér, ktorý sa snaží infikovať zariadenie, väčšinou využívaný na krádež citlivých údajov, najmä hesiel \cite{20}). 
\par Populárny prehliadač Google Chrome ponúka používateľovi možnosť zapamätať si heslá pri prihlasovaní na rôznych stránkach. Tieto heslá sú synchronizované do všetkých zariadení pomocou Google konta. Je ľahké sa dostať do tohto trezoru, stačí do URL text-boxu prehliadača napísať ,,chrome://settings/passwords'' \cite{21}. Zobrazí sa zoznam hesiel pre dané stránky, ktoré si prehliadač ukladal. Jedným tlačidlom sa odokryje heslo ako otvorený text. Chrome nevyžaduje žiadnu autentizáciu. Je jednoduché pre útočníka pri fyzickom prístupe k zariadeniu tieto dáta získať pár klikmi. Chrome vývojár Justin Schuh avšak argumentuje \cite{21}. Hovorí, že keď má niekto prístup do účtu operačného systému používateľa, vie, vidí a má prístup ku všetkému. Môžeme z tohto tvrdenia vyvodiť záver, že Chrome nevidí zmysel v chránení trezoru prehliadača (okrem hesla do Googlu účtu), lebo rozumie faktu, že ten, kto má prístup do operačného systému, má aj tak prístup ku všetkému.
\par Mozilla Firefox tiež nechráni svoj trezor hesiel, avšak narozdiel od Chromu ponúka aktiváciu master hesla. Spomeňme ešte Safari od Apple, ktorý chráni celý trezor heslom účtu operačného systému. Z výroku Schuha vyplýva, že takéto zabezpečenie je zbytočné. Apple pravdepodobne počíta s tým, že môže prísť k zneužitiu zariadenia už po prihlásení do systému, teda útočník nemusí poznať heslo. V takom prípade považujeme takéto zabezpečenie za múdre, avšak určite existujú bezpečnejšie spôsoby (za cenu komfortu).
\par Spomínaný prieskum z roku 2015 nepriniesol pozitívne výsledky. O tri roky neskôr sa uskutočnil ďalší \cite{22}. Spomedzi 2500 Američanov si 35\% nikdy heslá nemení. Robí tak iba po vyzvaní. Veľkým prekvapením bolo 11\% používateľov, ktorí si ich menia každý deň. The National Institute of Standards and Technology radí používateľom, aby si heslá menili nie pravidelne, ale až keď nastane ich prelomenie. Keď padla otázka, aký nástroj respondenti používajú na svoju ochranu na internete, víťazom bol antivírový software (53\%), password manager získal 24\%, čo je trojnásobný nárast za obdobie troch rokov\footnote{vychádzajúc zo vzorky opýtaných, teda určitá štatistická odchylka je pri týhto úvahách samozrejmosťou.}.
\par Napriek pozitívnemu nárastu popularity password managerov považujeme 24\% za malé číslo. Je pravdou, že tieto aplikácie sú na trhu nie tak dlho. Antivírový softvér má v tomto časový náskok. Tento softvér má avšak iný cieľ a zameranie v bezpečnosti, než password manager. Bolo by nezmyselné tieto dva nástroje na bezpečnosť porovnávať ako dve technologické riešenia, ktoré slúžia na rovnaký účel. U niektorých čitateľov možno vzniká otázka typu: \textit{Prečo používatelia nepoužívajú viac bezpečnostné nástroje na ochranu ich osobných údajov?} V druhom spomínanom prieskume z roku 2018 sa opýtaných pýtali aj na otázku, kedy boli poučení, respektíve vzdelaní na tému bezpečnosti na internete. Až 36\% nedostalo na túto tému žiadne vzdelanie. Aj toto môže byť odpoveďou na vyššie spomínanú otázku.
\par Ďalším dôvodom môže byť nedôvera. Používatelia nemusia byť presvedčení o tom, že password manager naozaj ich heslá ochráni, nezverejní, prípadne nezneužije. Jedna práca \cite{23} študovala, prečo má password manager stále málo používateľov. Urobila prieskum, kde prizvala 137 respondentov používajúcich password manager a 111 takých, ktorí ho nepoužívajú. Vznikla štúdia, ktorá porovnáva odpovede na 6 otázok týchto dvoch skupín a snaží sa vyvodiť záver, prečo je popularita týchto aplikácií taká, aká je.