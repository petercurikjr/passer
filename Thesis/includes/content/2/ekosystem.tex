\subsection{Ekosystém}
Vráťme sa ku kapitole \nameref{sucasnystavnatrhu}, kde sme rozoberali existujúcich password managerov. Pri ich podrobnej analýze sme dospeli k problému. 

Všetky z nich (pochopiteľne) uchovávajú a chránia heslá používateľa. To znamená, že sú prístupné iba na zariadení, ktoré v sebe obsahuje aplikáciu. Preto napríklad LastPass ponúka browser extension \cite{how_lastpass_works}. Používateľ si do internetového prehliadača môže nainštalovať rozšírenie. Prihlási sa so svojim LastPass účtom a tak môže k svojim heslám pristupovať. Teda, používateľ by mal LastPass nainštalovaný na svojom smartfóne, aj počítači s ľubovoľným prehliadačom. Toto mu uľahčí prístup k heslám.

Problémom je fakt, že používateľ potrebuje svoje heslá vždy, keď sa chce prihlásiť do nejakého z jeho účtov. Bez ohľadu na to, na akom zariadení prihlasovanie vykonáva. Zistili sme, že súčasné riešenia nedokážu pokryť tento problém v plnosti. Aj keď väčšina password managerov poskytuje podporu na rôznych zariadeniach (smartfóny, počítače, inteligentné hodinky...), väčšinou ide o niečo, čo používateľ vlastní a pravidelne využíva. 

Teda hľadáme odpoveď na otázku, čo má používateľ robiť, ak používa password manager, no potrebuje sa prihlásiť na cudzom zariadení. 

Priblížme si vyššie uvedenú problematiku na konkrétnom príklade. Majme používateľa password managera, ktorý sa momentálne nachádza na letisku. Svoju letenku si potrebuje vytlačiť z emailu na jednom z letiskových verejných počítačov. Môžeme už teraz povedať, že je malá pravdepodobnosť, že daný počítač použije niekedy opäť. Navyše, má málo času, pretože lietadlo odchádza o pár desiatok minút. Najprv sa musí prihlásiť do svojho emailu. Heslo si nepamätá, je príliš komplexné. Použije svoj smartfón s nainštalovaným password managerom, aby zistil, aké má heslo do svojho emailu. 

Teraz potrebuje ručne opísať používateľské meno a heslo do cudzieho počítača. Už počas tejto operácie vidíme niekoľko bezpečnostných hrozieb. Počas ručného opisovania je telefón príliš dlho vystavený krádeži. No možno ani nie je nutné ukradnúť samotný telefón. Útočníkovi stačí odfotiť obrazovku telefónu, prípadne si nenápadne heslo opísať na papier spoza chrbáta používateľa (toto je jednoduché docieliť, najmä ak je pred verejnými počítačmi rad ľudí alebo práve prechádza okolo veľký dav). 

Nehovoríme o bezpečnostnej hrozbe softvéru, ale o jeho praktickom používaní. Na prácu so svojimi účtami používajú používatelia väčšinou osobné zariadenie. Ale ak sa už stane, že musia určitý úkon s prihlasovaním vykonať na cudzom zariadení, v mnohých prípadoch to je na verejnosti (riziko napadnutia). Taktiež v mnohých prípadoch ide o časový zhon. V bežných situáciách nie je súčasťou plánu používať na osobné záležitosti cudzie zariadenie. Ručné prepisovanie hesla teda nie je len málo bezpečné, ale aj časovo nepraktické. Navyše, môže sa stať, že komplikované heslo používateľ správne prepíše až na niekoľký pokus. Pri zadávaní hesla je väčšinou heslo nahradené hviezdičkami, prípadne bodkami. Používateľ si nemusí všimnúť chybu pri písaní. Alebo si nevšimne, že je v počítači nastavený iný jazyk klávesnice. Na zahraničných letiskách môže byť nastavená anglická klávesnica. Tá má znaky \textit{z} a \textit{y} vymenené oproti slovenskej. Taktiež nie je potrebné držať klávesu \textit{Shift} pri písaní čísiel v hornej časti klávesnice. Špeciálne znaky majú tiež špecifickú klávesovú distribúciu v závislosti od jazyka klávesnice. Možností, ako nesprávne zadať heslo, je viacero. 

Existujú aj iné spôsoby, ako by používateľ na letisku mohol takúto situáciu riešiť. Ak jeho password manager podporuje aplikáciu pre operačný systém daného počítača, môže si ju stiahnuť a nainštalovať. Tu sa ale časová náročnosť oveľa viac preťahuje. Používateľ potrebuje aplikáciu vyhľadať na internete. Potom ju musí stiahnuť. Sťahovanie môže trvať dlho, ak je internetové pripojenie pomalé. Potom musí aplikáciu (prípadne browser extension) nainštalovať. Ak má počítač slabý výkon, táto operácia bude trvať o to dlhšie. Inštalácia sa ani nemusí podariť spustiť, kvôli právam v operačnom systéme. Musíme počítať s tým, že letisko mohlo zakázať inštalovať akýkoľvek softvér na ich počítačoch. To je pochopiteľné, nakoľko cudzí softvér môže obsahovať vírusy. Ak sa inštalácia podarí, používateľ sa do aplikácie musí prihlásiť, nájsť si heslo, ktoré práve potrebuje a použiť ho na prihlásenie. Vidíme tu náročný proces, ktorý nemusí byť praktický. 

Hovoríme o ekosystéme, ktorý password manager vytvára. Iba tam, kde je nainštalovaná aplikácia manager funguje. Mimo aplikácie je používateľ vo veľmi nepríjemných podmienkach. 