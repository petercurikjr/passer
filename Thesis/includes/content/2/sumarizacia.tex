\subsection{Sumarizácia}
Po analýze existujúcich riešení a mienke vzorky spoločnosti vieme veľa poznatkov. Existuje mnoho password managerov. Väčšina z nich je naozaj bezpečná a ponúka používateľovi mnohé možnosti. Táto technológia je avšak celkom nová. Správcovia hesiel sa (nielen) preto netešia vysokej popularite. Ďalšími dôvodmi môžu byť faktory ako: nevzdelanosť v danej problematike, nedôvera (aj vďaka rôznym bezpečnostným zlyhaniam populárnych password managerov), či fakt, že tieto aplikácie niektorí používatelia nepotrebujú alebo nepreferujú. Podľa prieskumov z rôznych rokov je avšak vidno, že ich popularita napriek všetkému rastie. Z tých, ktorí ho používajú na bežnej báze vieme, že kľúčová je pre nich pohodlnosť a bezpečnosť. 

Náš cieľ je po analytickej časti práce jasný. Potrebujeme password manager, ktorý je jendoduchý na používanie. Týmto zminimalizujeme odpor, strach, či úzkosť naučiť sa niečo nové. Toto uviedlo ako problém viacero respondentov zo skupiny nepoužívajúcich password manager. Taktiež potrebujeme password manager, ktorý je kvalitne zabezpečený. To je samozrejmosťou, nakoľko bezpečnosť je hlavným princípom takejto aplikácie. Z odpovedí používajúcich vieme, že oceňujú na prvom mieste pohodlnosť používania. Napriek tomu, že mimo ekosystému tento faktor dramaticky klesá. Preto sme presvedčení, že respondenti uviedli túto odpoveď vzhľadom k ich osobným zariadeniam, ktoré pravidelne využívajú a majú v nich nainštalovanú aplikáciu.

Ako sme spomínali v úvode práce, oblasť ekosystému password managerov považujeme za dostatočne rozvinutú. Toto potvrdzujú aj uvedené prieskumy. Respondentmi uvedené výhody ako bezpečnosť, či pohodlnosť používania ukazujú výborný stav password managerov súčasnosti. Preto sa ich ekosystému venovať nebudeme. 

Cieľom bakalárskej práce je tento ekosystém otvoriť. Vyvinúť password manager a nájsť bezpečný, pohodlný a efektívny spôsob prístupu používateľa k heslám daného password managera na cudzom zariadení. Výsledkom môže byť aj ukážka toho, že takéto niečo nie je možné. Je totiž pravdepodobné, že to už existujúce, veľké password manager spoločnosti skúšali, ale neuspeli. My si však myslíme, že nejaký spôsob existuje. No jeho zložitosť používania, časová zložitosť, prípadne iný faktor môžu byť horšie, než existujúce spôsoby (príklad z letiska: prepisovanie hesla z telefónu do cudzieho zariadenia pri prihlasovaní). V takom prípade nebudeme považovať výsledok tejto práce za riešenie.